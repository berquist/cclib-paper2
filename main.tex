%% ****** Start of file aiptemplate.tex ****** %
%%
%%   This file is part of the files in the distribution of AIP substyles for REVTeX4.
%%   Version 4.1 of 9 October 2009.
%%
%
% This is a template for producing documents for use with 
% the REVTEX 4.1 document class and the AIP substyles.
% 
% Copy this file to another name and then work on that file.
% That way, you always have this original template file to use.
\documentclass[aip,
jcp,
% bmf,
% sd,
% rsi,
 amsmath,amssymb,
 preprint,%
 % reprint,
 longbibliography]{revtex4-2}
%\documentclass[aip,reprint]{revtex4-1}
\usepackage[utf8]{inputenc}
\usepackage[T1]{fontenc}
\usepackage[version=4]{mhchem}
\usepackage{xcolor}
\usepackage{tabularx}
\usepackage{booktabs}
\usepackage{graphicx}
\usepackage{algorithm}
\usepackage{algpseudocode}
\usepackage{braket}
\usepackage{siunitx}
\usepackage{threeparttable}

\graphicspath{ {./images/} }

\newcommand{\citehere}{\textcolor{red}{\bf\textsuperscript{\textdagger}}}
\newcommand{\citeherecomment}[1]{\textcolor{red}{\bf\textsuperscript{\textdagger} #1}}

\definecolor{shiv_purple}{rgb}{0.6       ,  0.19607843,  0.8}
\definecolor{shiv_blue}{rgb}{0.11764706,  0.56470588,  1.}
\definecolor{shiv_green}{rgb}{0.        ,  0.57647059,  0.23529412}
\definecolor{shiv_yellow}{rgb}{0.97647059,  0.75686275,  0.1372549}
\definecolor{shiv_orange}{rgb}{1.        ,  0.54901961,  0.}
\definecolor{shiv_red}{rgb}{0.93333333,  0.20784314,  0.18039216}
\definecolor{shiv_gray}{rgb}{0.72156863,  0.71764706,  0.73333333}

\newcommand{\shiv}[1]{\textcolor{shiv_purple}{\bf [SU: #1 ]}}
\newcommand{\amanda}[1]{\textcolor{shiv_blue}{\bf [AD: #1 ]}}
\newcommand{\eric}[1]{\textcolor{shiv_green}{\bf [EB: #1 ]}}
\newcommand{\todo}[1]{\textcolor{shiv_red}{\bf [TO DO: #1 ]}}

\newcommand{\shivten}[5]{% #1 presuperscript, #2 presubscript, #3 tensor symbol, #4 postsuperscript, #5 postsubscript
    \sideset{^{#1}_{#2}}{^{#4}_{#5}}{\mathop{#3}}
}

\usepackage[acronym]{glossaries-extra}
\setabbreviationstyle[acronym]{long-short}

\glsdisablehyper

% \makeglossaries

%%%%%%%%%%%%%%%%%%%%%%%%%%%%%%%%%%%%%%%%%%%%%%%%%%%%%%%%%%%%%%%%%%%%%%%%%%%%%%%%%%%%%%%%%%%%%%%%%%%
% Chemical Systems
%%%%%%%%%%%%%%%%%%%%%%%%%%%%%%%%%%%%%%%%%%%%%%%%%%%%%%%%%%%%%%%%%%%%%%%%%%%%%%%%%%%%%%%%%%%%%%%%%%%
\newacronym{boa}{B.O. approximation}{Born-Oppenheimer approximation}
\newacronym{ueg}{UEG}{uniform electron gas}
%%%%%%%%%%%%%%%%%%%%%%%%%%%%%%%%%%%%%%%%%%%%%%%%%%%%%%%%%%%%%%%%%%%%%%%%%%%%%%%%%%%%%%%%%%%%%%%%%%%
% Wave Function
%%%%%%%%%%%%%%%%%%%%%%%%%%%%%%%%%%%%%%%%%%%%%%%%%%%%%%%%%%%%%%%%%%%%%%%%%%%%%%%%%%%%%%%%%%%%%%%%%%%
\newacronym[\glslongpluralkey={molecular orbitals},\glsshortpluralkey={MOs}]{mo}{MO}{molecular orbital}
\newacronym[\glslongpluralkey={atomic orbitals},\glsshortpluralkey={AOs}]{ao}{AO}{atomic orbital}
\newacronym{lcao}{LCAO}{linear combination of atomic orbitals}
\newacronym{hf}{HF}{Hartree-Fock}
\newacronym{scf}{SCF}{self-consistent field}
\newacronym{mf}{MF}{mean-field}
\newacronym{ci}{CI}{Configuration Interaction}
\newacronym{cis}{CIS}{Configuration Interaction Singles}
\newacronym{cisd}{CISD}{Configuration Interaction Singles and Doubles}
\newacronym{cisdt}{CISDT}{Configuration Interaction Singles, Doubles, and Triples}
\newacronym{cisdtq}{CISDTQ}{Coupled Cluster Singles, Doubles, Triples and Quadruples}
\newacronym{fci}{FCI}{Full Configuration Interaction}
\newacronym{cc}{CC}{Coupled Cluster}
\newacronym{ccsd}{CCSD}{Coupled Cluster Singles and Doubles}
\newacronym{ccsdt}{CCSDT}{Coupled Cluster Singles, Doubles, and Triples}
\newacronym{ccsdpt}{CCSD(T)}{Coupled Cluster Singles, Doubles, and Perturbative Triples}
\newacronym{eom}{EOM}{Equation-of-Motion}
\newacronym{casci}{CASCI}{Complete Active Space Configuration Interaction}
\newacronym{casscf}{CASSCF}{Complete Active Space Self-Consistent Field}
\newacronym{dlpno}{DLPNO}{Domain-based Local Pair Natural Orbital Coupled-Cluster}
%%%%%%%%%%%%%%%%%%%%%%%%%%%%%%%%%%%%%%%%%%%%%%%%%%%%%%%%%%%%%%%%%%%%%%%%%%%%%%%%%%%%%%%%%%%%%%%%%%%
% Density Functional Theory
%%%%%%%%%%%%%%%%%%%%%%%%%%%%%%%%%%%%%%%%%%%%%%%%%%%%%%%%%%%%%%%%%%%%%%%%%%%%%%%%%%%%%%%%%%%%%%%%%%%
\newacronym{dft}{DFT}{Density Functional Theory}
\newacronym{lda}{LDA}{local density approximation}
\newacronym{gga}{GGA}{generalized gradient approximation}
\newacronym{mgga}{mGGA}{meta-generalized gradient approximation}


\draft % marks overfull lines with a black rule on the right

\begin{document}
\title{\texttt{cclib} working title}

% JCP doesn't have a note about this footnote business but JAP does
%https://aip.scitation.org/jcp/info/policies
%https://aip.scitation.org/jap/info/policies
\author{Shiv Upadhyay}
\affiliation{University of Washington}
% add your name and affiliation here please
\date{\today}

% Use the \preprint command to place your local institutional report number 
% on the title page in preprint mode.
% Multiple \preprint commands are allowed.
%\preprint{}



% repeat the \author .. \affiliation  etc. as needed
% \email, \thanks, \homepage, \altaffiliation all apply to the current author.
% Explanatory text should go in the []'s, 
% actual e-mail address or url should go in the {}'s for \email and \homepage.
% Please use the appropriate macro for the type of information

% \affiliation command applies to all authors since the last \affiliation command. 
% The \affiliation command should follow the other information.


%\email[]{Your e-mail address}
%\homepage[]{Your web page}
%\thanks{}
%\altaffiliation{}


% Collaboration name, if desired (requires use of superscriptaddress option in \documentclass). 
% \noaffiliation is required (may also be used with the \author command).
%\collaboration{}
%\noaffiliation

\date{\today}


% \pacs{}% insert suggested PACS numbers in braces on next line

%\maketitle must follow title, authors, abstract and \pacs

% Body of paper goes here. Use proper sectioning commands. 
% References should be done using the \cite, \ref, and \label commands
\begin{abstract}
% motivation

% remaining problem to be address

% what we actually did

% what did we do 
% TODO edit here
Placeholder abstract

\end{abstract}
\maketitle 
\section{Introduction}

\citeherecomment{Everyone has a comment that will tag it with their initials \texttt{\textbackslash shiv (\shiv{}) and \textbackslash amanda (\amanda{}) and \textbackslash eric (\eric{}) and \textbackslash todo (\todo{})} please use these to leave a comment.}


\section{Theory}
\section{Computational Methods}


\section{Results \& Discussion}


\section{Conclusions}



\section*{Supplementary Material}


\section*{Acknowledgements}


%\bibliographystyle{IEEEtran}
\bibliography{references/cclib}


% If in two-column mode, this environment will change to single-column format so that long equations can be displayed. 
% Use only when necessary.
%\begin{widetext}
%$$\mbox{put long equation here}$$
%\end{widetext}

% Figures should be put into the text as floats. 
% Use the graphics or graphicx packages (distributed with LaTeX2e).
% See the LaTeX Graphics Companion by Michel Goosens, Sebastian Rahtz, and Frank Mittelbach for examples. 
%
% Here is an example of the general form of a figure:
% Fill in the caption in the braces of the \caption{} command. 
% Put the label that you will use with \ref{} command in the braces of the \label{} command.
%
% \begin{figure}
% \includegraphics{}%
% \caption{\label{}}%
% \end{figure}

% Tables may be be put in the text as floats.
% Here is an example of the general form of a table:
% Fill in the caption in the braces of the \caption{} command. Put the label
% that you will use with \ref{} command in the braces of the \label{} command.
% Insert the column specifiers (l, r, c, d, etc.) in the empty braces of the
% \begin{tabular}{} command.
%
% \begin{table}
% \caption{\label{} }
% \begin{tabular}{}
% \end{tabular}
% \end{table}

% If you have acknowledgments, this puts in the proper section head.
%\begin{acknowledgments}
% Put your acknowledgments here.
%\end{acknowledgments}

% Create the reference section using BibTeX:

\end{document}
%
% ****** End of file aiptemplate.tex ******
