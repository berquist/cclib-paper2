\documentclass[num-refs]{wiley-article}
\usepackage[utf8]{inputenc}
\usepackage[T1]{fontenc}
\usepackage[version=4]{mhchem}
\usepackage{xcolor}
\usepackage{tabularx}
\usepackage{booktabs}
\usepackage{graphicx}
\usepackage{algorithm}
\usepackage{algpseudocode}
\usepackage{braket}
\usepackage{siunitx}
\usepackage{threeparttable}

\graphicspath{ {./images/} }

\newcommand{\citehere}{\textcolor{red}{\bf\textsuperscript{\textdagger}}}
\newcommand{\citeherecomment}[1]{\textcolor{red}{\bf\textsuperscript{\textdagger} #1}}

\definecolor{shiv_purple}{rgb}{0.6       ,  0.19607843,  0.8}
\definecolor{shiv_blue}{rgb}{0.11764706,  0.56470588,  1.}
\definecolor{shiv_green}{rgb}{0.        ,  0.57647059,  0.23529412}
\definecolor{shiv_yellow}{rgb}{0.97647059,  0.75686275,  0.1372549}
\definecolor{shiv_orange}{rgb}{1.        ,  0.54901961,  0.}
\definecolor{shiv_red}{rgb}{0.93333333,  0.20784314,  0.18039216}
\definecolor{shiv_gray}{rgb}{0.72156863,  0.71764706,  0.73333333}

\newcommand{\shiv}[1]{\textcolor{shiv_purple}{\bf [SU: #1 ]}}
\newcommand{\amanda}[1]{\textcolor{shiv_blue}{\bf [AD: #1 ]}}
\newcommand{\eric}[1]{\textcolor{shiv_green}{\bf [EB: #1 ]}}
\newcommand{\todo}[1]{\textcolor{shiv_red}{\bf [TO DO: #1 ]}}

\newcommand{\shivten}[5]{% #1 presuperscript, #2 presubscript, #3 tensor symbol, #4 postsuperscript, #5 postsubscript
    \sideset{^{#1}_{#2}}{^{#4}_{#5}}{\mathop{#3}}
}

\usepackage[acronym]{glossaries-extra}
\setabbreviationstyle[acronym]{long-short}

\begin{document}
\maketitle
\selectlanguage{english}
%%%%%%%%%%%%%%%%%%%%%%%%%%%%%%%%%%%%%%%%%%%%%%%%%%%%%%%%%%%%%%%%%%%%%%%%%%%%%%%
% Abstract
%%%%%%%%%%%%%%%%%%%%%%%%%%%%%%%%%%%%%%%%%%%%%%%%%%%%%%%%%%%%%%%%%%%%%%%%%%%%%%%
\begin{abstract}
This is a LaTeX template designed for use by the \textbf{International Journal of Quantum Chemistry}. Please consult the journal's author guidelines in order to confirm that your manuscript complies with the journal's requirements. When you are ready to submit your manuscript, download/export it in LaTeX or Word format and submit your document at \href{https://mc.manuscriptcentral.com/qua}{https://mc.manuscriptcentral.com/qua}. Please replace this text with your abstract.

\textbf{Keywords} --- keyword 1, \emph{keyword 2}, keyword 3, keyword 4, keyword 5, keyword 6, keyword 7.
\end{abstract}

\section{Introduction}

cclib\cite{Oboyle2008} (\url{http://cclib.github.io/} and \url{https://github.com/cclib/cclib}) is an open-source library, written in Python, for parsing and interpreting the results of computational chemistry packages.  The goals of cclib are centered around the reuse of data obtained from these programs and contained in output files, specifically:
\begin{itemize}
\item extract (parse) data from the output files generated by multiple programs,
\item provide a consistent interface to the results of computational chemistry calculations, particularly those results that are useful for algorithms or visualization,
\item facilitate the implementation of algorithms that are not specific to a particular computational chemistry package, and
\item to maximize interoperability with other open source computational chemistry and cheminformatic software libraries.
\end{itemize}

In addition to following traditional semantic versioning practices, literature-citeable versions of cclib are also available through Zenodo, from 1.2\cite{Langner2014} through 1.7\cite{Berquist2021}, with a new citation for each minor release.

Anecdotally, a random sampling of the latest citations for the original paper shows that most citations are actually for GaussSum, one of the earliest tools that integrated cclib as a core component.

\section{Open development workflow and the cclib community}

TODO say something about Open Chemistry here?

\subsection{Contributing}

The primary driving force behind each of the author's individual contributions has been the need

\cite{Berquist2017,Upadhyay2020}

\subsection{Google Summer of Code}

Since 2015 (TODO), cclib has participated in Google Summer of Code (GSoC, \url{https://summerofcode.withgoogle.com/}), held between TODO and TODO each year.  Rather than be an individual organization, we apply under the Open Chemistry umbrella group, alongside Avogadro, Open Babel, DeepChem, RDKit, gnina, 3Dmol.js, NWChem, and Psi4.  Open Chemistry maintains a yearly list of project ideas, the latest iteration of which is located at \url{https://wiki.openchemistry.org/GSoC_Ideas_2020}.  cclib has gained a number of sizable contributions through GSoC which would have been time-consuming or infeasible with only the core team.

TODO put in paragraph form:
\begin{itemize}
\item 2015: did not participate
\item 2016: Support for both reading and writing CJSON (Sanjeed Schamnad)
\item 2017: Support for writing Molden and WFX files (Sagar Gaur); OpenChemVault, a framework for parsing outputs into a database and serving them with a web UI (\url{https://github.com/cclib/openchemvault}) (Nitish Garg)
\item 2018: New parsers for Molcas and Turbomole (Kunal Sharma)
\item 2019: did not participate
\item 2020: atomic partial charge methods (Minsik)
\end{itemize}

\section{Using cclib with the CLI/API interfaces}

\section{Highlighted developments in cclib 2.0}
\subsection{Parsers}

The programs currently supported along with their first version cclib ``knows'' about are ADF (since TODO), Dalton (since TODO), Firefly (since TODO), GAMESS (US) (since TODO), GAMESS-UK (since TODO), Gaussian (since TODO), Jaguar (since TODO), Molcas (since TODO), Molpro (since TODO), MOPAC (since TODO), NWChem (since TODO), ORCA (since TODO), Psi4 (since TODO), Q-Chem (since TODO), and Turbomole (since TODO).  Additionally, formatted checkpoint files (FChk or \texttt{fchk}) produced by Gaussian and Q-Chem can also be parsed, and a GAMESS \texttt{*.dat} parser has been requested.  Some programs produce even more output files by default, such as Molpro and especially Turbomole, requiring the parsing of multiple files and merging results into a single \texttt{ccData} object.

\subsection{Bridges}
    * New bridge to Atomic Simulation Environment (Felipe S. S. Schneider)
    * New bridge to PySCF (Amanda Dumi)
    * [GSOC2020] New bridge to Horton (Minsik Cho)
    * New bridge to Psi4 (Felipe S. S. Schneider)
\subsection{Methods}

    * [GSOC2020] New methods: Bader's QTAIM, Bickelhaupt, Stockholder, Hirshfeld, and DDEC6 partial charges (Minsik Cho)
    * [GSOC2020] Support reading cube files in volume method (Minsik Cho)

\subsection{Readers and writers}


\section{The future of cclib}

Despite the seemingly outdated idea of parsing human-readable text into machine-usable data,

large-scale workflows.\cite{Abbott2019,St.John2020}

TODO insert bar plot of citation counts of original paper for each year since publication

there is still a long tail of computation that is performed outside of JSON

the computational chemistry equivalent of the Protein Data Bank does not exist

like Open Babel and RDKit, which have organically grown into toolboxes covering large parts of cheminformatics space. Through its Python interface, RDKit in particular has its force fields integrated as part of QCEngine and is therefore generally available throughout the MolSSI ecosystem.

% \subsection{issues}
% \subsection{99  Future of nmo, nbasis and similar attributes feature, question}
% \subsection{131 Support fragment calculations in some way feature, parsers}
% \subsection{132 Support ONIOM calcs in some way Gaussian, feature}
% \subsection{180 Support Cfour output feature, parsers}
% \subsection{241 Strip output file timings Gaussian, beginner, question}
% \subsection{254 Parse GAMESS *.dat files GAMESS}
% \subsection{299 Implement gradients and Hessians parsers}
% \subsection{390 New attribute scfinfo for microiteration data. GAMESS, parsers, question}
% \subsection{419 Structure of future attributes feature, parsers, question}
% \subsection{474 Add finite difference gradients/Hessians to unit tests feature, tests}
% \subsection{518 Sunsetting Python2 maintenance, question}
% \subsection{537 Investigate initialization of logfile superclass maintenance, parsers}
% \subsection{612 Consider moving data notes into the code beginner, docs}
% \subsection{628 Rethink output of methods methods, question}
% \subsection{653 Natural Spin Orbitals parsing feature, parsers}
% \subsection{657 Support multistep jobs generally feature, parsers}
% \subsection{776 Replace periodictable and SciPy with QCElemental beginner, feature, maintenance}
% \subsection{890 Standardize Bridge (and methods?) API bridge, methods}
% \subsection{PRs}
% \subsection{781 Replace periodictable and SciPy with QCElemental}
% \subsection{798 QCSchema: add output writer}

\section*{Acknowledgements}
Acknowledgements should include contributions from anyone who does not meet the criteria for authorship (for example, to recognize contributions from people who provided technical help, collation of data, writing assistance, acquisition of funding, or a department chairperson who provided general support), as well as any funding or other support information.

\section*{Conflict of interest}
You may be asked to provide a conflict of interest statement during the submission process. Please check the journal's author guidelines for details on what to include in this section. Please ensure you liaise with all co-authors to confirm agreement with the final statement.
%
\FloatBarrier{}
\bibliography{bibliography.bib}
\end{document}
